\section{Descreva todos os tipos de seletores do CSS. Quais são as diferenças entre eles?}
Os seletores no CSS indicam ao navegador quais elementos HTML serão escolhidos para ter seus estilos modificados pelos valores informados pelos seletores. \\
Existem alguns tipos disponíveis para utilizar:
\begin{itemize}
  \item \textbf{de Tipo}: selecionam elementos HTML diretamente. É escrito como:
  \begin{verbatim}
    // 'div' é o elemento selecionado 
    div { 
      margin: "2rem";
    }
  \end{verbatim}
  \item \textbf{de Classe}: selecionam elementos HTML que contém valores específicos no atributo $class$. É escrito como:
  \begin{verbatim}
    // elementos com "class='header'" serão selecionados
    .header { 
      display: "flex";
    }
  \end{verbatim}
  \item \textbf{de Id}: seleciona um elemento HTML que contém um valor específico para o atributo $id$. É escrito como:
  \begin{verbatim}
    // o elemento com "id='logo'" será selecionado 
    #logo { 
      color: "blue"; 
    }
  \end{verbatim}
  \item \textbf{de Atributo}: seleciona somente elementos que contém valores atributos específicos. É escrito como:
  \begin{verbatim}
    // os inputs de tipo textual serão selecionados
    input[type="text"] { 
      border: none;
    }
  \end{verbatim}
  \item \textbf{de Pseudoclasses}: seleciona elementos que estão em um determinado estado naquele instante. É escrito como:
  \begin{verbatim}
    // quando o ponteiro do mouse passar por cima do botão (hover), 
    // esse estilo será aplicado
    button:hover { 
      color: "gray"
    }
  \end{verbatim}
  \item \textbf{de Combinadores}: seleciona elementos de outras formas não contempladas acima e misturando os seletores, como elementos filhos de outros, elementos com um tipo e classe específicas, e outros.
  \begin{verbatim}
    // elementos "li" que são filhos diretos de uma div 
    // com classe main serão selecionados
    div.main > li { 
      font-size: 12
    }
  \end{verbatim}
\end{itemize}
